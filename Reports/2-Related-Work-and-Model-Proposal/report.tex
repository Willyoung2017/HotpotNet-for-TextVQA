% File project.tex
%% Style files for ACL 2021
\documentclass[11pt,a4paper]{article}
\usepackage[hyperref]{acl2021}
\usepackage{times}
\usepackage{booktabs}
\usepackage{todonotes}
\usepackage{latexsym}
\renewcommand{\UrlFont}{\ttfamily\small}

% This is not strictly necessary, and may be commented out,
% but it will improve the layout of the manuscript,
% and will typically save some space.
\usepackage{microtype}

\aclfinalcopy 

\newcommand\BibTeX{B\textsc{ib}\TeX}

\title{11-777 Report 2: Related Work and Model Proposal}

\author{
  First Last 1\thanks{\hspace{4pt}Everyone Contributed Equally -- Alphabetical order} \hspace{2em} First Last 2$^*$ \hspace{2em} First Last 3$^*$ \hspace{2em} First Last 4$^*$ \\
  \texttt{\{ID1, ID2, ID3, ID4\}@andrew.cmu.edu}
  }

\date{}

\begin{document}
\maketitle
\section{Related Work and Background (5 papers per person)}
\paragraph{Related Datasets} 

\paragraph{Unimodal Baselines}

\paragraph{Prior Work}

\paragraph{Relevant techniques}


\clearpage
\section{Model Proposal (1 page)}
Include a diagram (e.g. labeled flow chart) of all modules.  This is not final!

\subsection{Overall model structure}

\subsection{Encoders}
Describe encoders for each modality and at least one alternatives for each.  Explain the relative strengths of each option (e.g. coverage, efficiency, ...)

\subsection{Decoders}

\subsection{Loss Functions}
Describe both your primary task loss and three possible auxiliary losses that might improve performance.  Justify your choices.

\clearpage
\section{Team member contributions}
\paragraph{Member 1} contributed ...

\paragraph{Member 2} contributed ...

\paragraph{...} contributed ...


% Please use 
\bibliographystyle{acl_natbib}
\bibliography{references}

%\appendix



\end{document}
